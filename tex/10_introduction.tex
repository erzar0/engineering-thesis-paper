\section{Introduction}
% \subsection{X-ray fluorescence data acquisition method}
\subsection{Motivation}
The work presented in this paper builds upon contributions made by dr. inż. Bartłomiej Łach in his doctoral thesis entitled \emph{Rozwój systemu detekcyjnego do obrazowania przestrzennego rozkładu pierwiastków metodą fluorescencji rentgenowskiej}\cite{Lach2022}  (eng. \emph{Development of a detection system for spatial imaging of element distribution using X-ray fluorescence}).

His dissertation was dedicated to development of system for non-invasive analysis of works of art that uses X-ray radiation to carry out measurements. 
The system presented in paper was designed to provide a map of element distribution in the top layers of the object, offering essential information for analyzing the pigments used in the artwork. 
The result would be important knowledge for conservators of monuments and art, informing them about the object's state of preservation, quality of past maintenance processes, the assessment of the work's authenticity, and the expansion of previous state of knowledge.

Most popular technique to achieve such results, namely MA-XRF (\textbf{MA}cro \textbf{X-R}ay \textbf{F}luorescence), operate by scanning selected parts of art point by point. 
This method is characterized by high spatial and energetic resolution. 
However, due to usage of polycapillary lenses it is limited to performing scans on 2D surfaces. 
Another drawback is long measurement time. 
For example, scanning a painting with an area of $65 \times 45$ $\text{cm}^{2}$, a resolution of $1300 \times 900$ pixels, and a  dwell time of $10$ ms per pixel could take about $3.5$ hours.\cite{Alfeld2013}. 

An alternative method involves scanning not just a single point but multiple points within a specific area of the surface, utilizing FF-XRF (\textbf{F}ull-\textbf{F}ield \textbf{X-R}ay \textbf{F}luorescence). 
This approach was implemented in the DETART (\textbf{DET}ector for \textbf{ART}) data acquisition system. 
DETART not only can scan large portion of the surface (over $10e5$ pixels at the same time), but it is also capable of performing scans of 3D objects without losing spatial resolution thanks to near infinite depth of field provided by pinhole camera. 

After data acquisition part, the analysis can be performed. 
Łach proposed three different methods for analyzing data: "standard" ROI (\textbf{R}egion \textbf{O}f \textbf{I}nterest) method and two different machine learning algorithms: PCA (\textbf{P}rincipal \textbf{C}omponents \textbf{A}nalysis) and NFM (\textbf{N}on-negative \textbf{M}atrix \textbf{F}actorization). 
In this paper, a method based on deep learning instead will be proposed and tested. 

\subsection{Concept}
Concept of this thesis is heavily inspired by the work presented in \cite{Jones2022}. 
The authors of this article introduced the idea of a CNN (\textbf{C}onvolutional \textbf{N}eural \textbf{N}etwork) model tasked with classifying every sample measured with XRF as one pigment class. 
The model was initially trained on synthetic data generated by software, achieving an accuracy of 55\% when tested on real XRF spectra, and then 96\% after applying transfer learning using small quantity of real XRF samples.  
This approach has several advantages, such as full automation of the process, and high accuracy after applying transfer learning.
However, there are some downsides to consider:
\begin{itemize}
    \item The model can only classify pigments it has learned about
    \item It does not provide any information about signals coming from elements, which may offer some valuable insights about the studied object
\end{itemize}
A potential solution to these problems may be to shift the focus from classifying pigments to classifying elements. A multi-label classification approach might be used to assess existence of each learned element independently from the tested pigment. This approach is more similar to algorithms used in Łach's thesis.
