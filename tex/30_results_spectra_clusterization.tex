\section{Results}
\subsection{Spectra Clusterization}
Clusterization of spectrum will be presented using data gathered on ``Portrait of John III Sobieski in Karacena Scale Armour'' (see \prettyref{fig:sobieski_fragment}).
\begin{figure}[H] 
  \centering     
  \includegraphics[width=0.5\textwidth]{img/jan_sobieski_w_karacenie.png} 
  \caption{An examined fragment of ``Portrait of John III Sobieski in Karacena Scale Armour''. Source: \cite{wikimediaSobieskiPortrai} }
  \label{fig:sobieski_fragment}
\end{figure}

\subsubsection{Clusterization With SOM}
The first clustering algorithm, SOM, was invoked as presented in \prettyref{lst:som-invocation}.
\newenvironment{longlistingG}{\captionsetup{type=listing, width=0.8\textwidth}}{}
\begin{longlistingG}
    \pythoncode{listings/som_invocation.py}
    \caption{Invocation of SOM algorithm}
    \label{lst:som-invocation}
\end{longlistingG}
\vspace{12pt}

The arguments used to initialize \texttt{SelfOrganizingMap} are:
\begin{description}
    \item[\texttt{input\_size}:] Configured to the number of channels (4094 in this instance).
    \item[\texttt{map\_size}:] Set to (3, 3), indicating a total of $3 \times 3 = 9$ clusters.
    \item[\texttt{learning\_rate}:] Represents the learning rate during the first epoch.
    \item[\texttt{sigma\_neigh}:] Sets the influence on the neighbors of the BMU. Lower values result in larger influence.
    \item[\texttt{sigma\_decay}:] Sets the rate of decay for the learning rate. Lower values result in slower decay.
\end{description}
The performance of the algorithm scales linearly with the number of clusters. 
With a total of 9 clusters, the clusterization of data with shape of (524, 348, 4094) takes approximately 6 minutes. 
Therefore, it may not be feasible method if many more clusters were needed, at least while using basic python implementation.

The result of applying clusterization on unchanged spectrum can be observed on \prettyref{fig:sobieski_clustered_som_noise}.

\newpage
\begin{figure}[H] 
  \centering     
  \includesvg[width=0.8\textwidth]{img/sobieski_clustered_som_noise.svg} 
  \caption{Clusterization of unprocessed spectrum of ``Portrait of John III Sobieski in Karacena Scale Armour'' using SOM}
  \label{fig:sobieski_clustered_som_noise}
\end{figure}

As the reader may notice, certain clusters are just artifacts. 
It was sensible to investigate the possibility of these artifacts arising from a poor implementation of the SOM algorithm (although it was not the best first course of action). 
This involved employing a UMAP to HDBSCAN pipeline to validate the clustering results.

\subsubsection{Clusterization With UMAP And HDBSCAN}
Clusterization with UMAP and HDBSCAN was invoked as presented in \prettyref{lst:umap-hdbscan-invocation}.

\newenvironment{longlistingH}{\captionsetup{type=listing, width=0.8\textwidth}}{}
\begin{longlistingH}
    \pythoncode{listings/umap_hdbscan_invocation.py}
    \caption{Invocation of UMAP to HDBSCAN pipeline}
    \label{lst:umap-hdbscan-invocation}
\end{longlistingH}
\vspace{12pt}

The meaning of arguments passed to UMAP is as follows:
\begin{description}
    \item[n\_neighbors:] The number of neighbors used for manifold approximation. As \texttt{n\_neighbors} increases more focus is placed on the global structure of the data. 
    \item[min\_dist:] The minimum distance between points in the dimensional embedding. For clusterization, values around 0 are preferred, because it is desirable to transform data into "tightly packed" clumps.
    \item[n\_components:] The number of dimensions in the low-dimensional representation. Experimentation showed that reducing dimensions to (only!) one gave the best results. 
\end{description}

Arguments for HDBSCAN are:
\begin{description}
    \item[min\_samples:] The number of samples in a neighborhood for a point to be considered a core point.
    \item[min\_cluster\_size:] The minimum number of points required to form a cluster. It is probably the most important parameter; smaller values will lead to more clusters.
    \item[metric:] The distance metric used for clustering. Euclidean distance works well, although other popular metrics also give good results.
\end{description}

Dimensionality reduction using UMAP with the presented arguments is quite costly. 
The cost scales with the parameter \texttt{n\_neighbors} and, in this case, takes approximately 7 minutes. 
Further clusterization with HDBSCAN is much faster (due to the really low dimensionality of input) and allows for fast experiments with different parameters. 
It is useful mainly to achieve the desired number of clusters.

It appears that although clusterization worked, it did not prevented artifacts from appearing - 
\begin{figure}[H] 
  \centering     
  \includesvg[width=0.8\textwidth]{img/sobieski_clustered_hdbscan_noise.svg} 
  \caption{Clusterization of unprocessed spectrum of ``Portrait of John III Sobieski in Karacena Scale Armour'' using UMAP and HDBSCAN}
  \label{fig:sobieski_clustered_som_noise}
\end{figure}