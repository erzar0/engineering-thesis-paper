\section{Methodology}
\subsection{Used Technologies}

All code developed during the project was written in the \texttt{python3} programming language. Google Colab was utilized as runtime environment, as it provides resources that allows rapid testing of DNN (Deep Neural Network) implementations. 
Google Colab also enabled easy sharing of interactive notebooks with the thesis supervisor.

The primary technologies and libraries used in the project include:
\begin{itemize}
    \item \texttt{python3} programming language.
    \item \texttt{pytorch} - Chosen as the deep learning framework due to the author's familiarity with it and its widespread adoption among researchers (over 60\% of new paper implementations use \texttt{pytorch} \cite{papersWithCodeTrends}).
    \item \texttt{hdbscan} - Implementation of the HDBSCAN (Hierarchical Density-Based Spatial Clustering of Applications with Noise) algorithm for data clusterization.
    \item \texttt{umap-learn} - Implementation of the UMAP (Uniform Manifold Approximation and Projection) algorithm for dimensionality reduction.
    \item \texttt{minisom} - Implementation of a SOM (Self-Organizing Map) algorithm for data clusterization.
    \item \texttt{scikit-learn}, \texttt{scipy}, \texttt{matplotlib}, \texttt{pandas}, \texttt{numpy} - Common tools used for data analysis in the \texttt{python} ecosystem.
\end{itemize}
