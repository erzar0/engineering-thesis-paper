\begin{abstract}
During recent years, deep learning models have proven their usability in numerous tasks that cannot be easily modeled using classic methods. 
In the meantime, a novel method for non-invasive XRF (X-ray fluorescence ) data acquisition has been developed. 
Although algorithms with proven efficacy for analyzing XRF data already exist, and unlike deep learning models, they are easily explainable, they also have shortcomings. 
For instance, the commonly used ROI (Region of Interest) technique encounters difficulties in detecting elements invisible in accumulated spectra, differentiating signals with very similar characteristic energies and obtaining information about their actual distribution in the signal. 
This paper presents an attempt to create a ViT (Vision Transformer) model for multi label classification of chemical elements present in spectra measured with XRF.
\end{abstract}