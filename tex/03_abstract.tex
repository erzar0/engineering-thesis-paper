\begin{abstract}
During recent years, deep learning models have proven their usability in numerous tasks that cannot be easily modeled using linear methods. The range of application areas begins with school projects, such as creating a binary classification model to differentiate between photos of cats and dogs, and extends to advanced applications like autonomous vehicles and Artificial General Intelligence (AGI)-like Large Language Models (LLM). In the field of material sciences, a novel method for non-invasive X-ray fluorescence (XRF) data acquisition has been developed. Although algorithms with proven efficacy for analyzing these spectra already exist, and unlike deep learning models, they are easily explainable, they also have shortcomings. For instance, the commonly used Region of Interest (ROI) technique encounters difficulties in detecting elements invisible in accumulated spectra, differentiating signals with very similar characteristic energies and obtaining information about their actual distribution in the signal. This paper presents an attempt to create a convolution neural network (CNN) based model for classifying chemical elements present in spectra acquired using X-ray fluorescence techniques. 
\end{abstract}